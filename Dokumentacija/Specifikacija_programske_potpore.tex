\chapter{Specifikacija programske potpore}
		
	\section{Funkcionalni zahtjevi}
			
			\textbf{\textit{dio 1. revizije}}\\
			
			\textit{Navesti \textbf{dionike} koji imaju \textbf{interes u ovom sustavu} ili  \textbf{su nositelji odgovornosti}. To su prije svega korisnici, ali i administratori sustava, naručitelji, razvojni tim.}\\
				
			\textit{Navesti \textbf{aktore} koji izravno \textbf{koriste} ili \textbf{komuniciraju sa sustavom}. Oni mogu imati inicijatorsku ulogu, tj. započinju određene procese u sustavu ili samo sudioničku ulogu, tj. obavljaju određeni posao. Za svakog aktora navesti funkcionalne zahtjeve koji se na njega odnose.}\\
			
			
			\noindent \textbf{Dionici:}
			
			\begin{packed_enum}
				
				\item Dionik 1
				\item Dionik 2				
				\item ...
				
			\end{packed_enum}
			
			\noindent \textbf{Aktori i njihovi funkcionalni zahtjevi:}
			
			
			\begin{packed_enum}
				\item  \underbar{Aktor 1 (inicijator) može:}
				
				\begin{packed_enum}
					
					\item funkcionalnost 1
					\item funkcionalnost 2
					\begin{packed_enum}
						
						\item  podfunkcionalnost 1 
						\item  podfunkcionalnost 2
				
					\end{packed_enum}
					\item  funkcionalnost 3
					
				\end{packed_enum}
			
				\item  \underbar{Aktor 2 (sudionik) može:}
				
				\begin{packed_enum}
					
					\item funkcionalnost 1
					\item funkcionalnost 2
					
				\end{packed_enum}
			\end{packed_enum}
			
			\eject 
			
			
				
			\subsection{Obrasci uporabe}
				
				\textbf{\textit{dio 1. revizije}}
				
				\subsubsection{Opis obrazaca uporabe}
					\textit{Funkcionalne zahtjeve razraditi u obliku obrazaca uporabe. Svaki obrazac je potrebno razraditi prema donjem predlošku. Ukoliko u nekom koraku može doći do odstupanja, potrebno je to odstupanje opisati i po mogućnosti ponuditi rješenje kojim bi se tijek obrasca vratio na osnovni tijek.}\\
					

					\noindent \underbar{\textbf{UC1 - Prijava}}
					\begin{packed_item}
	
						\item \textbf{Glavni sudionik: } Administrator
						\item  \textbf{Cilj:} Prijava administratora u sustav
						\item  \textbf{Sudionici:} Administrator i baza podataka
						\item  \textbf{Preduvjet:} Nema
						\item  \textbf{Opis osnovnog tijeka:}
						
						\item[] \begin{packed_enum}
	
							\item Otvaranje aplikacije unutar web preglednika
							\item Unos korisničkog imena i lozinke
							\item Podnošenje zahtjeva za prijavu klikom na gumb
							\item Korisnik biva preusmjeren na početnu stranicu
						\end{packed_enum}
						
						\item  \textbf{Opis mogućih odstupanja:}
						
						\item[] \begin{packed_item}
	
							\item[2.a] Uneseni podatci ne odgovaraju traženom formatu
							\item[] \begin{packed_enum}
								
								\item Ispis upozorenja o krivom formatu i onemogućen gumb za prijavu sve dok podatci ne zadovoljavaju traženi format
								
							\end{packed_enum}

							\item[4.a] Korisnički podatci su neispravni ili nisu prepoznati u bazi podataka
							\item[] \begin{packed_enum}
								
								\item Korisnika ne preusmjeravamo na početnu stranicu već mu samo ispisujemo da prijava nije uspjela.
								
							\end{packed_enum}
						\end{packed_item}
					\end{packed_item}
					
					%------------------------------------------------------------------------------
					\noindent \underbar{\textbf{UC2 -Dodavanje novog administratora}}
					\begin{packed_item}
						
						\item \textbf{Glavni sudionik: }Smještajni administrator
						\item  \textbf{Cilj:} Dodati korisničke podatke novog administratora i dodijeliti mu odgovarajuće uloge
						\item  \textbf{Sudionici:} Smještajni administrator i baza podataka
						\item  \textbf{Preduvjet:} UC1: Prijava
						\item  \textbf{Opis osnovnog tijeka:}
						
						\item[] \begin{packed_enum}
							
							\item Administrator odabire opciju za dodavanje novog administratora
							\item Unosi korisničke podatke novog administratora
							\item Označuje uloge dodijeljene novom administratoru
							\item Podnosi zahtjev za unosom novog administratora u bazu podataka
							\item Sva polja se postavljaju na početne vrijednosti
							
						\end{packed_enum}
						
						\item  \textbf{Opis mogućih odstupanja:}
						
						\item[] \begin{packed_item}
							
							\item[2.a] Uneseni podatci ne odgovaraju traženom formatu
							\item[] \begin{packed_enum}
								
								\item Ispis upozorenja o krivom formatu i onemogućen gumb za dodavanje novog administratora sve dok podatci ne zadovoljavaju traženi format
								
							\end{packed_enum}

							\item[3.a] Nije označena ni jedna uloga
							\item[] \begin{packed_enum}
								
								\item Onemogućen gumb za dodavanje novog administratora sve dok nije označena barem jedna uloga novog administratora
								
							\end{packed_enum}
							
							\item[4.a] Sustav vraća grešku prilikom dodavanja novog administratora
							\item[] \begin{packed_enum}
								
								\item Ispisati tekst greške
								\item Čekati na novi pokušaj podnošenja zahtjeva(Korak 4.)
								
							\end{packed_enum}
							
						\end{packed_item}
					\end{packed_item}
					
					%------------------------------------------------------------------------------
					\noindent \underbar{\textbf{UC3 -Unos raspoloživog smještaja}}
					\begin{packed_item}
						
						\item \textbf{Glavni sudionik: }Smještajni administrator
						\item  \textbf{Cilj:} Unos novog smještaja u bazu podataka
						\item  \textbf{Sudionici:} Smještajni administrator i baza podataka
						\item  \textbf{Preduvjet:} UC1: Prijava
						\item  \textbf{Opis osnovnog tijeka:}
						
						\item[] \begin{packed_enum}
							
							\item Administrator odabire opciju za dodavanje novog smještaja
							\item Odabir tipa stana
							\item Odabir kategorije stana 
							\item Unos maksimalnog kapaciteta stana
							\item Unos adrese na kojoj se stan nalazi(UC4)
							\item Unos podataka o zgradi(Broj kata, broj stana, dostupnost lifta, opis)
							\item Odabir tipa vlasništva
							
						\end{packed_enum}
						
						\item  \textbf{Opis mogućih odstupanja:}
						
						\item[] \begin{packed_item}
							
							\item[2.a] Nije odabran ni jedan tip stana
							\item[] \begin{packed_enum}
								
								\item Onemogućeno podnošenje zahtjeva za dodavanjem smještaja u bazu
								\item Ispis upozorenja o odabiru
								
							\end{packed_enum}
							
							\item[3.a] Nije odabran ni jedna kategorija stana
							\item[] \begin{packed_enum}
								
								\item Onemogućeno podnošenje zahtjeva za dodavanjem smještaja u bazu
								\item Ispis upozorenja o odabiru
								
							\end{packed_enum}
							
							\item[4.a] Uneseni podatak nije u odgovarajućem formatu
							\item[] \begin{packed_enum}
								
								\item Onemogućeno podnošenje zahtjeva za dodavanjem smještaja u bazu
								\item Ispis greške o krivom formatu
								
							\end{packed_enum}
							
							\item[7.a] Odabran tip privatnog vlasništva
							\item[] \begin{packed_enum}
								
								\item Unos podataka vezanih za stan u vlasništvu(UC3.1)
								
							\end{packed_enum}
							
							\item[7.b] Odabran tip stana u najmu
							\item[] \begin{packed_enum}
								
								\item Unos podataka vezanih za stan u najmu(UC3.2)
								
							\end{packed_enum}
							
						\end{packed_item}
					\end{packed_item}
					
					%------------------------------------------------------------------------------
					\noindent \underbar{\textbf{UC3.1 -Smještaj u vlasništvu}}
					\begin{packed_item}
						
						\item \textbf{Glavni sudionik: }Smještajni administrator
						\item  \textbf{Cilj:} Unos podataka o smještaju u vlasništvu
						\item  \textbf{Sudionici:} Smještajni administrator i baza podataka
						\item  \textbf{Preduvjet:} UC1: Prijava i UC3: Unos raspoloživog smještaja
						\item  \textbf{Opis osnovnog tijeka:}
						
						\item[] \begin{packed_enum}
							
							\item Podnošenje zahtjeva za dodavanje smještaja u bazu podataka
							\item Preusmjeravanje na početnu stranicu
							
						\end{packed_enum}
						
						\item  \textbf{Opis mogućih odstupanja:}
						
						\item[] \begin{packed_item}
							
							\item[1.a] Dogodila se greška prilikom dodavanja smještaja
							\item[] \begin{packed_enum}
								
								\item Ispis teksta greške administratoru
								\item Povratak na 1. korak
								
							\end{packed_enum}
							
						\end{packed_item}
					\end{packed_item}
					
					%------------------------------------------------------------------------------
					\noindent \underbar{\textbf{UC3.2 -Smještaj u najmu}}
					\begin{packed_item}
						
						\item \textbf{Glavni sudionik: }Smještajni administrator
						\item  \textbf{Cilj:} Unos podataka o smještaju u najmu
						\item  \textbf{Sudionici:} Smještajni administrator i baza podataka
						\item  \textbf{Preduvjet:} UC1: Prijava i UC3: Unos raspoloživog smještaja
						\item  \textbf{Opis osnovnog tijeka:}
						
						\item[] \begin{packed_enum}
							
							\item Unos vremena dostupnosti stana
							\item Podnošenje zahtjeva za dodavanje smještaja u bazu podataka
							\item Preusmjeravanje na početnu stranicu
							
						\end{packed_enum}
						
						\item  \textbf{Opis mogućih odstupanja:}
						
						\item[] \begin{packed_item}
							
							\item[1.a] Nije uneseno vrijeme dostupnosti
							\item[] \begin{packed_enum}
								
								\item Onemogućeno podnošenje zahtjeva za dodavanje smještaja u bazi
								\item Ispis greške o dostupnosti
								
							\end{packed_enum}
							
							\item[2.a] Dogodila se greška prilikom dodavanja smještaja
							\item[] \begin{packed_enum}
								
								\item Ispis teksta greške administratoru
								\item Povratak na 2. korak
								
							\end{packed_enum}
							
						\end{packed_item}
					\end{packed_item}
					
					%------------------------------------------------------------------------------
					\noindent \underbar{\textbf{UC4 -Odabir lokacije}}
					\begin{packed_item}
						
						\item \textbf{Glavni sudionik: }Smještajni administrator
						\item  \textbf{Cilj:} Unos podataka o lokaciji i prikaz lokacije na karti
						\item  \textbf{Sudionici:} Smještajni administrator i \textit{Google Maps}
						\item  \textbf{Preduvjet:} UC1: Prijava
						\item  \textbf{Opis osnovnog tijeka:}
						
						\item[] \begin{packed_enum}
							
							\item Unos podataka o lokaciji(Grad, ulica, kućni broj)
							\item Prikaz unesene lokacije na krati pomoću servisa \textit{Google Maps}
							
						\end{packed_enum}
						
						\item  \textbf{Opis mogućih odstupanja:}
						
						\item[] \begin{packed_item}
							
							\item[1.a] Podatci ne odgovaraju traženom formatu
							\item[] \begin{packed_enum}
								
								\item Ne upućujemo zahtjev na \textit{Google Maps}
								\item Onemogućen odabir lokacije
								\item Ispis greške u podacima
								
							\end{packed_enum}
							
							\item[2.a] Dogodila se greška prilikom prikaza lokacije
							\item[] \begin{packed_enum}
								
								\item Onemogućen odabir lokacije
								\item Ispis greške dobivene od \textit{Google Mapsa}
								
							\end{packed_enum}
							
						\end{packed_item}
					\end{packed_item}
				
					
				\subsubsection{Dijagrami obrazaca uporabe}
					
					\textit{Prikazati odnos aktora i obrazaca uporabe odgovarajućim UML dijagramom. Nije nužno nacrtati sve na jednom dijagramu. Modelirati po razinama apstrakcije i skupovima srodnih funkcionalnosti.}
				\eject		
				
			\subsection{Sekvencijski dijagrami}
				
				\textbf{\textit{dio 1. revizije}}\\
				
				\textit{Nacrtati sekvencijske dijagrame koji modeliraju najvažnije dijelove sustava (max. 4 dijagrama). Ukoliko postoji nedoumica oko odabira, razjasniti s asistentom. Uz svaki dijagram napisati detaljni opis dijagrama.}
				\eject
	
		\section{Ostali zahtjevi}
		
			\textbf{\textit{dio 1. revizije}}\\
		 
			 \textit{Nefunkcionalni zahtjevi i zahtjevi domene primjene dopunjuju funkcionalne zahtjeve. Oni opisuju \textbf{kako se sustav treba ponašati} i koja \textbf{ograničenja} treba poštivati (performanse, korisničko iskustvo, pouzdanost, standardi kvalitete, sigurnost...). Primjeri takvih zahtjeva u Vašem projektu mogu biti: podržani jezici korisničkog sučelja, vrijeme odziva, najveći mogući podržani broj korisnika, podržane web/mobilne platforme, razina zaštite (protokoli komunikacije, kriptiranje...)... Svaki takav zahtjev potrebno je navesti u jednoj ili dvije rečenice.}
			 
			 
			 
	