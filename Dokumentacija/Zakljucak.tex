\chapter{Zaključak i budući rad}
		
		%\textbf{\textit{dio 2. revizije}}\\
		
		 %\textit{U ovom poglavlju potrebno je napisati osvrt na vrijeme izrade projektnog zadatka, koji su tehnički izazovi prepoznati, jesu li riješeni ili kako bi mogli biti riješeni, koja su znanja stečena pri izradi projekta, koja bi znanja bila posebno potrebna za brže i kvalitetnije ostvarenje projekta i koje bi bile perspektive za nastavak rada u projektnoj grupi.}
		 
		 Zadatak naše grupe je bio stvoriti web aplikaciju "DentAll" koja bi omogućila tvrtkama koje omogućuju medicinskim turistima prijevoz i smještaj da vide i organiziraju te podatke. Nakon 12 tjedana rada, zadatak je djelomično ispunjen. Moguće je dodavati nove elemente preko backenda u bazu podataka, ali to nije moguće učiniti s frontenda. Sustav na frontendu ispisuje upisane korisnike, vozače i smještaj, ali ne prikazuje upisane adminstratore. Moguće je otvoriti formu za upis novih administratora, vozača, smještaja, ali nije moguće otvoriti formu za unos novog korisnika. Nije moguće promijeniti upisane podatke niti ih isbrisati. %provjeri ovo ponovno!!!
		 \newline Projekt se može podjeliti na dvije faze. U prvoj fazi radili smo na podjeli rada, u početku smo imali samostalnu podjelu zadataka, a nakon otprilike tri tjedna rada konačno smo se podjelili u tri podtima, podtim za frontend koji je brojio tri člana, podtim za backend i bazu podataka koji su oba brojila po dva člana. Prva faza završava prvom predajom projekta.
		 \newline
		 U drugoj fazi ostaje podjela na tri podtima s tim da je podtim koji je radio na bazi podatka preuzmu sav rad na dokumentaciji. Naglasak u ovoj fazi bio je na implementaciji projekta, a faza je trajala do konačne predaje projekta.
		 \newline
		 Aplikaciju je moguće proširit na mnoge načine, jedna ideja bila bi omogućiti samim korisnicima zdrastvenih usluga da mogu provjeriti informacije o svome posjetu. Rad na projektu dakako je vrijedno iskustvo za sve članove tima, svima nama je ovo bio prvi puta rad na ovakvoj aplikaciji i u ovakovom formatu, te se većina nas do sad nije susrela s tehnologijama poput LaTex-a i Git-a. Konflikata u timu nije bilo, ali nedostatak koordinatora i neiskustvo voditelja dovelo je do ponekad manjka komunikacije među članovima i konačno djelomično ispunjenje zadatka.
		
		 %\textit{Potrebno je točno popisati funkcionalnosti koje nisu implementirane u ostvarenoj aplikaciji.}
		
		\eject 