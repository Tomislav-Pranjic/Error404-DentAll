\chapter{Opis projektnog zadatka}
		
		
		{Cilj ovog projekta razvoj je aplikacije za evidenciju i koordinaciju smještaja i prijevoza korisnika usluga zdravstvenog turizma. U današnje vrijeme turizam te sve aktivnosti vezane uz turizam sve su više prisutne u većini država. Mnogi ljudi svojevoljno odlaze u druge države kako bi odradili određeni medicinski postupak, bilo to zbog manje cijene, bolje usluge ili nekog drugog razloga. Ogroman broj klinika pokušava pronaći više klijenta čak i preko granice. Veliki broj stomatoloških klinika u urbanim centrima kao i na Jadranskoj obali povećava obim posla oglašavanjem u inozemstvu i pružanjem usluga strancima. \\ Ovom aplikacijom pomoglo bi se ne samo turistima, kojima je potrebna usluga zdravstvenog turizma, već i domaćim klinikama kojima bi se obujam posla povećao, što bi značilo i mogućnost za veće plaće zaposlenicima i/ili zapošljavanjem većeg broja ljudi. U aplikaciji bi bila ugrađena kompletna organizacija prijevoza i smještaja, što bi znatno povećalo privlačnost produkta krajnjim korisnicima, a povećala bi se kompetitivnost u odnosu na druge pružatelje sličnih usluga.}\\
		
		{Danas možemo vidjeti kako je zdravstveni turizam prisutan u gotovo svakoj državi. Neki od primjera su medicinska rehabilitacija u poznatim toplicama BiH, Spa u Belgiji, razna lječilišta termalni vodama Italije, no isto tako postoje brojni primjeri i u Hrvatskoj. Samo jedan od primjera u Hrvatskoj su stomatološke ordinacije na Jadranu.}\\
		
		{Hrvatska privlači turiste zbog svojih ljekovitih termalnih izvora u Toplicama, slikovitih wellness odmarališta na obali Jadranskog mora te renomiranih stomatoloških klinika u Zagrebu. Izvan Hrvatske, Turska nudi putnicima ljekovite termalne kupke u Pamukkaleu, dok Tajland impresionira posjetitelje svojim spa centrima na tropskim otocima, pružajući nezaboravna iskustva u zdravstvenom turizmu.}\\
		
		{Korisnici koji bi mogli biti zainteresirani za ovu uslugu su npr. ljudi kojima je potrebna određena zdravstvena usluga kao što su wellness, toplice, stomatološka usluga itd., a koje bi privlačila niža cijena tretmana nego u njihovim državama ili bi jednostavno htjeli razgledati ljepote Hrvatske te usput obaviti zdravstvenu uslugu koja im je potrebna. }\\
		
		{Aplikacija koja na izbor daje sve moguće prijevoznike, podatke, rutu putovanja, raspoloživi smještaj i još mnogo toga uvelike bi pojednostavila sam postupak planiranja putovanja korisnicima zdravstvenog turizma. Sve što je potrebno je prijaviti se na aplikaciju, odabrati datum i vrijeme planiranog dolaska, a aplikacija bi se pobrinula za sve ostalo. U par koraka možete se riješiti problema smještaja i prijevoza koji su Vam potrebni kako bi medicinska usluga protekla na najbolji mogući način. }
		{U aplikaciji postoje tri uloge:}
		
		\begin{packed_item}
			\item Smještajni administrator
			\item Administrator prijevoznih usluga
			\item Korisnički administrator
		\end{packed_item}
		
		{\textbf{\textit{Smještajni administrator}} unosi podatke o smještaju koji je u to vrijeme raspoloživ. Svaka klinika ima raspoložive smještajne objekte, bilo u najmu ili privatnom vlasništvu. Ova vrsta administratora može mijenjati sve što se tiče samog smještaja, od kapaciteta do vlasništva samog smještajnog objekta. Isto tako smještajni administrator može i obrisati smještajnu jedincu. Podaci smještaja sastoje se od:}
		\begin{packed_item}
			\item Tipa stana
			\begin{packed_item}
				\item Veličina i vlasništvo
			\end{packed_item}
			
			\item Kategorije opremljenosti
			\begin{packed_item}
				\item Broj zvjezdica (1 – najslabije opremljen do 5- najbolje opremljen)
				\item Također je opisano sve što stan dodatno posjeduje
			\end{packed_item}
			
			\item Adrese
			\begin{packed_item}
				\item Adresa zgrade, kat, broj stana
			\end{packed_item}
			
			\item Vlasništva
			\begin{packed_item}
				\item Privatno ili u najmu
			\end{packed_item}
			
			\item Vremenski period dostupnosti za korištenje
		\end{packed_item}		
		
		
		{Također je omogućen grafički prikaz geografskog položaja nekretnine korištenjem Google Maps usluge.}
		
		{Uloga \textbf{\textit{administratora prijevoznih usluga}} u aplikaciji regulacija je samih prijevoznika. Podatci koje unosi u aplikaciju su:}
		\begin{packed_item}
			\item {osobni podaci prijevoznika: ime, prezime, broj telefona}
			\item {tip vozila, kapacitet te registracijska tablica}
			\item {radno vrijeme u kojem je prijevoznik dostupan}
		\end{packed_item}
		{Prijevoznici se mogu brisati, a njihovi neosnovni podaci mijenjati (kao što su radno vrijeme te radni dani u tjednu).}\\
		
		
		{Za definiciju korisnika medicinske usluge zaslužan je \textbf{\textit{korisnički administrator}} koji  unosi njihove osobne podatke te podatke bitne za rad. Ime, prezime i osnovni kontaktni podaci korisnika, vrijeme i mjesto dolaska i odlaska u/iz zemlje te preferencije vezano uz veličinu i kvalitetu smještaja. \\Detalji njihovih tretmana ne unose se ručno već se aplikacijskim sučeljem preuzimaju iz aplikacije za evidenciju medicinskih usluga. Sučelje je realizirano kao umjetno ispitno sučelje, a podaci o tretmanima popunjavaju se korištenjem isitnih primjera. \\Korisničkom administratoru iznimno je važno vrijeme i mjesto dolaska određenog korisnika kao i vrsta smještaja za koju je sam korisnik zainteresiran.\\Ova vrsta administratora može definirati ostale korisnike te im davati određene uloge.}
		{Jednom kada se svi podaci unesu u aplikaciju, aplikacija predlaže određenu smještajnu jedinicu te označava je zauzetom u tom vremenskom periodu.}\\
		
		{Aplikacija periodički provjerava status unosa medicinskih termina komunikacijom s aplikacijom medicinskih usluga. Ako se primi odgovor o zaključanom planu medicinskih usluga s listom medicinskih termina, potrebno je  pridijeliti raspoložive prijevoznike svakom od termina (prijevoz od smještaja u ordinaciju te povratak) te označiti prijevoznike zauzetima u tim terminima. }
		
		{Nakon što je ukupan plan završen, šalje se poruka elektroničke pošte korisniku medicinske usluge podacima ukupnog plana njihovog puta uključujući podatke o prijevozima i smještaju. Također se svakom od prijevoznika šalju posebne poruke s kontaktnim podacima korisnika kao i vremenima i adresama smještaja korisnika.}\\
		
		
		{Ova aplikacija mogla bi se nadograditi i unaprijediti u različitim smjerovima. Neke od ideja za unapređenje aplikacije su:}
		\begin{packed_item}
			\item Implementacija i unapređenje aplikacije za mobilne uređaje (primarno je samo za ekrane računala) uz pomoć responzivnog dizajna u HTML-u
			
			\item Nakon provedenog boravka u nekom od smještaja, korisnici mogu dati određenu ocjenu smještaju i/ili prijevozniku te uz ocjenu ostaviti neobavezni odgovarajući komentar
			
			\item U aplikaciji postaviti mogućnost kartičnog plaćanja pri rezervaciji određene zdravstvene usluge
			
			\item Mogućnost virtualnog obilaska smještaja prije nego što se klijenti odluče za rezervaciju baš tog smještaja
			
			\item Dodavanje opcije za razgled grada u kojem su smješteni (glavne ulice, posebne znamenitosti i slično)
			
			\item Unaprijediti web stranicu na više različitih jezika, kako je aplikacija namijenjena za korisnike zdravstvenog turizma to će biti bolje što je više jezika implementirano u aplikaciji
		\end{packed_item}
		
		
		
		
		
		
		
	